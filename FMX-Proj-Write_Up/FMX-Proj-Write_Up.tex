\documentclass[11pt,preprint, authoryear]{elsarticle}

\usepackage{lmodern}
%%%% My spacing
\usepackage{setspace}
\setstretch{1.2}
\DeclareMathSizes{12}{14}{10}{10}

% Wrap around which gives all figures included the [H] command, or places it "here". This can be tedious to code in Rmarkdown.
\usepackage{float}
\let\origfigure\figure
\let\endorigfigure\endfigure
\renewenvironment{figure}[1][2] {
    \expandafter\origfigure\expandafter[H]
} {
    \endorigfigure
}

\let\origtable\table
\let\endorigtable\endtable
\renewenvironment{table}[1][2] {
    \expandafter\origtable\expandafter[H]
} {
    \endorigtable
}


\usepackage{ifxetex,ifluatex}
\usepackage{fixltx2e} % provides \textsubscript
\ifnum 0\ifxetex 1\fi\ifluatex 1\fi=0 % if pdftex
  \usepackage[T1]{fontenc}
  \usepackage[utf8]{inputenc}
\else % if luatex or xelatex
  \ifxetex
    \usepackage{mathspec}
    \usepackage{xltxtra,xunicode}
  \else
    \usepackage{fontspec}
  \fi
  \defaultfontfeatures{Mapping=tex-text,Scale=MatchLowercase}
  \newcommand{\euro}{€}
\fi

\usepackage{amssymb, amsmath, amsthm, amsfonts}

\def\bibsection{\section*{References}} %%% Make "References" appear before bibliography


\usepackage[round]{natbib}

\usepackage{longtable}
\usepackage[margin=2.3cm,bottom=2cm,top=2.5cm, includefoot]{geometry}
\usepackage{fancyhdr}
\usepackage[bottom, hang, flushmargin]{footmisc}
\usepackage{graphicx}
\numberwithin{equation}{section}
\numberwithin{figure}{section}
\numberwithin{table}{section}
\setlength{\parindent}{0cm}
\setlength{\parskip}{1.3ex plus 0.5ex minus 0.3ex}
\usepackage{textcomp}
\renewcommand{\headrulewidth}{0.2pt}
\renewcommand{\footrulewidth}{0.3pt}

\usepackage{array}
\newcolumntype{x}[1]{>{\centering\arraybackslash\hspace{0pt}}p{#1}}

%%%%  Remove the "preprint submitted to" part. Don't worry about this either, it just looks better without it:
\makeatletter
\def\ps@pprintTitle{%
  \let\@oddhead\@empty
  \let\@evenhead\@empty
  \let\@oddfoot\@empty
  \let\@evenfoot\@oddfoot
}
\makeatother

 \def\tightlist{} % This allows for subbullets!

\usepackage{hyperref}
\hypersetup{breaklinks=true,
            bookmarks=true,
            colorlinks=true,
            citecolor=blue,
            urlcolor=blue,
            linkcolor=blue,
            pdfborder={0 0 0}}


% The following packages allow huxtable to work:
\usepackage{siunitx}
\usepackage{multirow}
\usepackage{hhline}
\usepackage{calc}
\usepackage{tabularx}
\usepackage{booktabs}
\usepackage{caption}


\newenvironment{columns}[1][]{}{}

\newenvironment{column}[1]{\begin{minipage}{#1}\ignorespaces}{%
\end{minipage}
\ifhmode\unskip\fi
\aftergroup\useignorespacesandallpars}

\def\useignorespacesandallpars#1\ignorespaces\fi{%
#1\fi\ignorespacesandallpars}

\makeatletter
\def\ignorespacesandallpars{%
  \@ifnextchar\par
    {\expandafter\ignorespacesandallpars\@gobble}%
    {}%
}
\makeatother

\newenvironment{CSLReferences}[2]{%
}

\urlstyle{same}  % don't use monospace font for urls
\setlength{\parindent}{0pt}
\setlength{\parskip}{6pt plus 2pt minus 1pt}
\setlength{\emergencystretch}{3em}  % prevent overfull lines
\setcounter{secnumdepth}{5}

%%% Use protect on footnotes to avoid problems with footnotes in titles
\let\rmarkdownfootnote\footnote%
\def\footnote{\protect\rmarkdownfootnote}
\IfFileExists{upquote.sty}{\usepackage{upquote}}{}

%%% Include extra packages specified by user

%%% Hard setting column skips for reports - this ensures greater consistency and control over the length settings in the document.
%% page layout
%% paragraphs
\setlength{\baselineskip}{12pt plus 0pt minus 0pt}
\setlength{\parskip}{12pt plus 0pt minus 0pt}
\setlength{\parindent}{0pt plus 0pt minus 0pt}
%% floats
\setlength{\floatsep}{12pt plus 0 pt minus 0pt}
\setlength{\textfloatsep}{20pt plus 0pt minus 0pt}
\setlength{\intextsep}{14pt plus 0pt minus 0pt}
\setlength{\dbltextfloatsep}{20pt plus 0pt minus 0pt}
\setlength{\dblfloatsep}{14pt plus 0pt minus 0pt}
%% maths
\setlength{\abovedisplayskip}{12pt plus 0pt minus 0pt}
\setlength{\belowdisplayskip}{12pt plus 0pt minus 0pt}
%% lists
\setlength{\topsep}{10pt plus 0pt minus 0pt}
\setlength{\partopsep}{3pt plus 0pt minus 0pt}
\setlength{\itemsep}{5pt plus 0pt minus 0pt}
\setlength{\labelsep}{8mm plus 0mm minus 0mm}
\setlength{\parsep}{\the\parskip}
\setlength{\listparindent}{\the\parindent}
%% verbatim
\setlength{\fboxsep}{5pt plus 0pt minus 0pt}



\begin{document}



\begin{frontmatter}  %

\title{Automated Forecasting Dashboards -- The Case of CPI}

% Set to FALSE if wanting to remove title (for submission)




\author[Add1]{Jan-Hendrik Pretorius\footnote{\textbf{Contributions:}
  \newline \emph{The authors would like to thank Codera Analytics for
  access to their data through EconData. Thank you sincerely.}}}
\ead{janhpret@gmail.com}

\author[Add2]{Nico Katzke}
\ead{nfkatzke@gmail.com}




\address[Add1]{Stellenbosch University, Stellenbosch, South Africa}
\address[Add2]{Satrix, Cape Town, South Africa}

\cortext[cor]{Corresponding author: Jan-Hendrik Pretorius\footnote{\textbf{Contributions:}
  \newline \emph{The authors would like to thank Codera Analytics for
  access to their data through EconData. Thank you sincerely.}}}

\begin{abstract}
\small{
In this study, I employ a sophisticated forecasting methodology to
project future values of the Consumer Price Index (CPI) for a specific
time series. The approach leverages the power of two distinct models: an
ARIMA (AutoRegressive Integrated Moving Average) model, adept at
capturing underlying trends and patterns, and a GARCH (Generalized
Autoregressive Conditional Heteroskedasticity) model, which excels in
modeling volatility and uncertainty. The ARIMA model is utilized to
generate point forecasts for CPI values, providing valuable insights
into the expected future trajectory of the index. Concurrently, the
GARCH model is employed to forecast volatility, allowing one to gauge
the level of uncertainty associated with the CPI projections. Through
this synthesis process, I combine the strengths of both models,
producing a comprehensive forecasting framework. This framework equips
one with not only CPI point forecasts but also their associated
confidence intervals, providing a holistic view of the anticipated CPI
trends and the inherent uncertainty in the forecast.
}
\end{abstract}

\vspace{1cm}


\begin{keyword}
\footnotesize{
Multivariate GARCH \sep Kalman Filter \sep Copula \\
\vspace{0.3cm}
}
\footnotesize{
\textit{JEL classification} L250 \sep L100
}
\end{keyword}



\vspace{0.5cm}

\end{frontmatter}

\setcounter{footnote}{0}


\renewcommand{\contentsname}{Table of Contents}
{\tableofcontents}

%________________________
% Header and Footers
%%%%%%%%%%%%%%%%%%%%%%%%%%%%%%%%%
\pagestyle{fancy}
\chead{}
\rhead{}
\lfoot{}
\rfoot{\footnotesize Page \thepage}
\lhead{}
%\rfoot{\footnotesize Page \thepage } % "e.g. Page 2"
\cfoot{}

%\setlength\headheight{30pt}
%%%%%%%%%%%%%%%%%%%%%%%%%%%%%%%%%
%________________________

\headsep 35pt % So that header does not go over title




\hypertarget{introduction}{%
\section{\texorpdfstring{Introduction
\label{Intro}}{Introduction }}\label{introduction}}

In recent years, South Africa has witnessed significant volatility in
its Consumer Price Index (CPI). The CPI is not only a fundamental
measure for assessing inflation and economic health but also plays a
pivotal role in finance, influencing monetary policy, investment
strategies, and financial planning. In the realm of finance, the CPI
directly impacts interest rate decisions, affects the value of currency,
and serves as a critical gauge for adjusting the real returns on
investments. As such, volatility in the CPI can introduce substantial
uncertainty into financial markets, complicating the assessment of
investment risks and opportunities.

This volatility poses a challenge for accurate forecasting and
necessitates innovative approaches capable of navigating the complexity
and unpredictability of financial indicators. Traditional forecasting
models often struggle to separate the underlying signal from the noise
inherent in such volatile economic data, leading to forecasts that may
be less reliable for economic planning, policy-making, and financial
analysis. Recognizing this challenge, this paper introduces an Automated
Forecasting Dashboard --
\href{https://janpretorius.shinyapps.io/incast/}{InCast}
(\textbf{In}flation Fore\textbf{cast}er) -- designed specifically to
enhance the precision and reliability of CPI forecasts in South Africa.
By employing advanced statistical models, including Autoregressive
Integrated Moving Average (ARIMA) and Generalized Autoregressive
Conditional Heteroskedasticity (GARCH) models, the dashboard aims to
refine our understanding of CPI trends and their implications for the
financial sector.

This paper details the development and implementation of the forecasting
model and its integration into an automated dashboard, emphasizing its
potential to significantly impact financial decision-making in South
Africa.

\hypertarget{literature}{%
\section{\texorpdfstring{Literature
\label{Lit}}{Literature }}\label{literature}}

In the literature review exploring the optimal modeling procedure for
capturing volatility in CPI rates, two significant methodologies are
highlighted: ARIMA/ARFIMA and GARCH modeling. The ARIMA model excels at
capturing the linear aspects of time series data, making it particularly
effective for modeling and forecasting CPI rates over short to medium
terms. On the other hand, the GARCH model is adept at capturing and
modeling the volatility clustering often observed in financial time
series, including CPI rates. This ability allows GARCH models to provide
valuable insights into the variability and risk associated with
inflation rates, offering a complementary perspective to the
ARIMA/ARFIMA models' focus on the level and trend of the series.
Together, these methodologies provide a comprehensive toolkit for
analyzing CPI volatility, with each addressing different facets of the
data's behavior.

A working paper by
(\protect\hyperlink{ref-Nyoni2019}{\textbf{Nyoni2019?}}) utilized ARIMA
models to forecast CPI in Myanmar using annual data from 1960 to 2017.
The ARIMA (2, 2, 1) model was identified as optimal for predicting
future CPI values
(\protect\hyperlink{ref-Nyoni2019}{\textbf{Nyoni2019?}}), indicating an
upward trajectory in Myanmar's inflation over the next decade. This
research emphasizes the model's stability and acceptability for CPI
modeling, advocating for policymakers to adopt stringent monetary and
fiscal policies to manage inflation effectively.

(\protect\hyperlink{ref-Boateng2016}{\textbf{Boateng2016?}}) explore the
long memory dynamics of CPI inflation rates in Ghana using ARFIMA
(AutoRegressive Fractionally Integrated Moving Average) modeling. This
methodology effectively captures both short- and long-term dependencies
within the inflation data, indicating persistence and mean-reversion in
inflation rates
(\protect\hyperlink{ref-Boateng2016}{\textbf{Boateng2016?}}). Their
findings underscore the significance of employing fractional
differencing to accurately model inflationary trends, offering vital
insights for policymakers. This approach not only reveals the presence
of long memory in Ghana's inflation but also aids in better
understanding and forecasting inflationary pressures
(\protect\hyperlink{ref-Boateng2016}{\textbf{Boateng2016?}}), proving
ARFIMA's effectiveness in economic time series analysis.

In his seminal work,
(\protect\hyperlink{ref-Bollerslev1986}{\textbf{Bollerslev1986?}})
introduced the GARCH model, extending the ARCH model developed by
(\protect\hyperlink{ref-Engle1982}{\textbf{Engle1982?}}). Bollerslev's
GARCH model allows for both autoregressive and moving average components
in the variance equation, enabling a more comprehensive analysis of
time-series volatility (see p.25). This model has become fundamental in
financial econometrics for analyzing and forecasting time-varying
volatility, reflecting its capacity to capture the clustering of
volatility phenomena observed in financial market returns.

Through the integration of ARIMA and GARCH methodologies, researchers
have adeptly navigated the complexities of CPI rate volatility,
achieving precise volatility modeling and signal extraction. This blend
harnesses ARIMA's strength in understanding data trends and GARCH's
prowess in volatility dynamics, offering a robust framework for accurate
economic forecasting.

(\protect\hyperlink{ref-Baillie1996}{\textbf{Baillie1996?}})
investigates inflation through the ARFIMA-GARCH model, highlighting its
effectiveness in capturing long-memory processes and conditional
heteroscedasticity in inflation data. Their methodology, combining
ARFIMA to model fractional integration and GARCH(1, 1) for volatility,
provides a nuanced understanding of inflation's persistence and
variability. This approach is particularly adept at analyzing the
complex dynamics of inflation rates, offering significant insights into
their mean-reverting behavior and the interaction between mean and
volatility, in line with Friedman's hypothesis that current inflation
significantly influences volatility in future inflation
(\protect\hyperlink{ref-Baillie1996}{\textbf{Baillie1996?}}).

In an extension of the ARFIMA-GARCH model,
(\protect\hyperlink{ref-Belkhouja2016}{\textbf{Belkhouja2016?}}) account
for time-varying baseline mean and volatility in analyzing G7 inflation
dynamics from 1955 to 2014. Their innovative approach, incorporating
logistic functions for structural changes (see p.451), addresses the
overestimation of long-run and GARCH persistence when such changes are
ignored. This model provides a nuanced understanding of inflation's
behavior, aligning identified shifts with significant economic and
political events, thus offering a more accurate tool for policymakers.

The ARIMA model is adept at capturing and forecasting the underlying
patterns in CPI data, while the GARCH model addresses the volatility of
residuals from the ARIMA model, a critical aspect given the data's
inherent noise and volatility. This sophisticated modeling approach,
combined with the dashboard's automation features---such as automated
data updating---ensures that forecasts are based on the most current
data available, a vital requirement for making informed financial
decisions. The paper now turns to the methodology followed to build an
automated forecasting dashboard.

\hypertarget{data-methodology}{%
\section{\texorpdfstring{Data \& Methodology
\label{Meth}}{Data \& Methodology }}\label{data-methodology}}

For the purposes of this essay, the focus will fall on using an
ARIMA-GARCH model\footnote{While the ARFIMA model is clearly more adept
  at handling inflation forecasting, I focus on ARIMA due to its
  simplistic nature. Future analyses might benefit from the ARFIMA
  methodology.} to extract the signal from Consumer Price Index (CPI)
data in South Africa. In this section, we will provide an overview of
the data sources, the key variables involved, and the methodology
employed for the analysis. This approach aims to uncover valuable
insights into the behavior and volatility of CPI, a critical economic
indicator, in the South African context.

\hypertarget{automated-data-processes}{%
\subsection{Automated Data Processes}\label{automated-data-processes}}

To conduct this analysis, I leverage an extensive dataset, produced by
Statistics South Africa, comprising the Classification of Individual
Consumption by Purpose (COICOP) 5-digit CPI values for South Africa.
This dataset is sourced directly from Codera Analytics's
\href{https://www.econdata.co.za/app}{EconData} platform
(\protect\hyperlink{ref-data}{Statistics South Africa, 2024}). It spans
the time frame from January 2008 to December 2023.

One distinct advantage of utilizing EconData is its capacity for
functional coding, which allows us to access the data seamlessly. This
functional coding approach renders the data entirely self-contained,
eliminating the necessity for separate data storage. Furthermore, it
offers the benefit of automated data updates on a monthly basis. Thus,
each time the dashboard (which I will elaborate on later) is accessed,
it provides the most up-to-date CPI information available. This
real-time data accessibility is instrumental in ensuring the accuracy
and timeliness of our analysis.

The central focus of this paper revolves around the South African CPI
for all items. However, it is worth noting that I also explore
additional facets of the CPI, as elaborated in Table
\ref{tab1}\footnote{Here I present the variable descriptions and the
  identifier linked to each variable on the EconData platform}. These
supplementary CPI components are considered in the context of
forecasting, and their analysis is facilitated through the utilization
of the \href{https://janpretorius.shinyapps.io/incast/}{InCast} web
application, which serves as a complement to this paper.

\begingroup\fontsize{12pt}{13pt}\selectfont
\begin{longtable}{ll}
\caption{CPI Code Lists as Supplied by EconData 
                \label{tab1}} \\ 
  \toprule
"CPI - All items" & 00.0.0.0.TC \\ 
  \hline 
\endhead 
\hline 
{\footnotesize Continued on next page} 
\endfoot 
\endlastfoot 
 \midrule
"CPI - Food and non alcoholic beverages" & 01.0.0.0.TC \\ 
  "CPI - Alcoholic beverages and tobacco" & 02.0.0.0.TC \\ 
  "CPI - Clothing and footwear" & 03.0.0.0.TC \\ 
  "CPI - Housing water electricity gas and other fuels" & 04.0.0.0.TC \\ 
  "CPI - Furnishings household equipment and routine household maintenance" & 05.0.0.0.TC \\ 
  "CPI - Health" & 06.0.0.0.TC \\ 
  "CPI - Transport" & 07.0.0.0.TC \\ 
  "CPI - Communication" & 08.0.0.0.TC \\ 
  "CPI - Recreation and culture" & 09.0.0.0.TC \\ 
  "CPI - Education" & 10.0.0.0.TC \\ 
  "CPI - Restaurants and hotels" & 11.0.0.0.TC \\ 
  "CPI - Miscellaneous goods and services" & 12.0.0.0.TC \\ 
   \bottomrule
\end{longtable}
\endgroup

\hypertarget{arima-and-garch-implementation}{%
\subsection{ARIMA and GARCH
Implementation}\label{arima-and-garch-implementation}}

The analysis in this essay is based on the integration of two prominent
time series modeling techniques: AutoRegressive Integrated Moving
Average (ARIMA) and Generalized Autoregressive Conditional
Heteroskedasticity (GARCH). ARIMA models are well-suited for capturing
the temporal dependencies and trends within time series data, while
GARCH models excel in modeling volatility and conditional
heteroskedasticity.

Our approach involves a two-step process:

\begin{enumerate}
\def\labelenumi{\arabic{enumi}.}
\tightlist
\item
  ARIMA Modeling: In the first step, we apply ARIMA modeling to the CPI
  data to extract the underlying trends and seasonality patterns. This
  step helps us understand the inherent dynamics of CPI and identify any
  significant autocorrelations or non-stationarity in the series.
\end{enumerate}

Given the time series CPI data sequence \((Y_t)\), the ARIMA(p,d,q)
model, can be expressed as:

\[
(1 - \phi_1 L - \phi_2 L^2 - \ldots - \phi_p L^p)(1 - L)^d Y_t = (1 + \theta_1 L + \theta_2 L^2 + \ldots + \theta_q L^q) \epsilon_t 
\]

Where:

\begin{itemize}
\tightlist
\item
  \(Y_t\) represents the observed value at time \(t\).
\item
  \(L\) is the lag operator, which shifts the time index back by one
  step.
\item
  \(p\) is the order of the autoregressive (AR) component.
\item
  \(d\) is the degree of differencing applied to make the series
  stationary.
\item
  \(q\) is the order of the moving average (MA) component.
\item
  \(\phi_1, \phi_2, \ldots, \phi_p\) are the autoregressive
  coefficients.
\item
  \(\theta_1, \theta_2, \ldots, \theta_q\) are the moving average
  coefficients.
\item
  \(\epsilon_t\) represents white noise, assumed to be independent and
  identically distributed with mean zero and constant variance.
\end{itemize}

This expression represents the general form of an ARIMA model, and the
specific coefficients (\(\phi\) and \(\theta\)) are then estimated.

\begin{enumerate}
\def\labelenumi{\arabic{enumi}.}
\setcounter{enumi}{1}
\tightlist
\item
  GARCH Modeling: Following the ARIMA analysis, we implement a GARCH
  model to examine the volatility and conditional heteroskedasticity in
  the CPI series. That is, I use the residuals from the ARIMA process as
  an input into a GARCH model. This step allows us to assess the degree
  of uncertainty and risk associated with inflation in South Africa.
  Given a time series of squared residuals \((\epsilon_t^2)\), a
  GARCH(1,1) model can be expressed as:
\end{enumerate}

For each time series \(Y_t\) of the residuals calculated from the ARIMA
model, I specify the GARCH(1,1) model as follows:

\[
   \sigma_t^2 = \omega + \alpha_1 \epsilon_{t-1}^2 + \beta_1 \sigma_{t-1}^2
   \]

Where: - \(\sigma_t^2\) is the conditional variance at time \(t\). -
\(\omega\) is the constant term representing the long-term average of
the conditional variance. - \(\alpha_1\) is the autoregressive parameter
for the conditional variance. - \(\epsilon_{t-1}^2\) is the squared
residual error from the previous time step. - \(\beta_1\) is the moving
average parameter for the conditional variance. - \(\sigma_{t-1}^2\) is
the conditional variance from the previous time step.

I then fit a GARCH(1,1) model for each time series \(Y_t\) using the
specified GARCH model parameters.

\hfill

\begin{enumerate}
\def\labelenumi{\arabic{enumi}.}
\setcounter{enumi}{2}
\tightlist
\item
  Synthesis: The synthesis of the ARIMA and GARCH models involves
  combining the point forecasts from the ARIMA model with the volatility
  forecasts from the GARCH model:
\end{enumerate}

Let \(Y_t\) represent the observed values of the Consumer Price Index
(CPI) time series.

\begin{itemize}
\tightlist
\item
  The ARIMA model provides point forecasts for \(Y_t\) as
  \(\(\hat{Y}_t\)\).
\item
  The GARCH model provides forecasts of conditional variance or
  volatility, denoted as \(\(\hat{\sigma}_t\)\), for each time step in
  the forecast horizon.
\item
  The CPI forecast at time \(t\) is denoted as \(\(\hat{Y}_t\)\).
\item
  The upper bound of the 95\% confidence interval for \(\(\hat{Y}_t\)\)
  is denoted as \(\(\hat{Y}_{t, \text{Upper}}\)\).
\item
  The lower bound of the 95\% confidence interval for \(\(\hat{Y}_t\)\)
  is denoted as \(\(\hat{Y}_{t, \text{Lower}}\)\).
\end{itemize}

Mathematically, the synthesis can be represented as follows:

{[} \hat{Y}\_t = \text{ARIMA Forecast} \textbackslash{}

\hat{\sigma}\_t = \text{GARCH Forecast} \textbackslash{} {]}

\#The 95\% confidence interval for \(\(\hat{Y}_t\)\) can be calculated
as:

{[} \hat{Y}\_\{t, \text{Upper}\} = \hat{Y}\emph{t + z}\{\alpha/2\}
\cdot \hat{\sigma}\_t \textbackslash{}

\hat{Y}\_\{t, \text{Lower}\} = \hat{Y}\emph{t - z}\{\alpha/2\}
\cdot \hat{\sigma}\_t \textbackslash{} {]}

Where: - \(\hat{Y}_t\) is the point forecast for CPI at time \(t\) from
the ARIMA model. - \(\hat{\sigma}_t\) is the volatility forecast at time
\(t\) from the GARCH model. - \(z_{\alpha/2}\) is the critical value of
the standard normal distribution corresponding to the desired confidence
level (e.g., \(z_{0.025}\) for a 95\% confidence interval).

By combining these two modeling techniques, we aim to gain a
comprehensive understanding of CPI behavior in South Africa, enabling us
to make informed observations and draw meaningful conclusions about
inflation dynamics in the region.

\hypertarget{automated-dashboard-design}{%
\subsection{Automated Dashboard
Design}\label{automated-dashboard-design}}

\hypertarget{results}{%
\section{\texorpdfstring{Results
\label{Results}}{Results }}\label{results}}

\includegraphics{FMX-Proj-Write_Up_files/figure-latex/unnamed-chunk-2-1.pdf}

\hypertarget{discussion-conclusion}{%
\section{Discussion \& Conclusion}\label{discussion-conclusion}}

\includegraphics{FMX-Proj-Write_Up_files/figure-latex/unnamed-chunk-3-1.pdf}

Blah blah

\includegraphics{FMX-Proj-Write_Up_files/figure-latex/model_garch-1.pdf}

blah blah

\includegraphics{FMX-Proj-Write_Up_files/figure-latex/unnamed-chunk-4-1.pdf}

\hypertarget{conclusion}{%
\section{Conclusion}\label{conclusion}}

\newpage

\hypertarget{references}{%
\section*{References}\label{references}}
\addcontentsline{toc}{section}{References}

\hypertarget{refs}{}
\begin{CSLReferences}{1}{0}
\leavevmode\vadjust pre{\hypertarget{ref-data}{}}%
Statistics South Africa. 2024.

\end{CSLReferences}

\hypertarget{appendix}{%
\section*{Appendix}\label{appendix}}
\addcontentsline{toc}{section}{Appendix}

\hypertarget{appendix-a}{%
\subsection*{Appendix A}\label{appendix-a}}
\addcontentsline{toc}{subsection}{Appendix A}

Some appendix information here

\hypertarget{appendix-b}{%
\subsection*{Appendix B}\label{appendix-b}}
\addcontentsline{toc}{subsection}{Appendix B}

\bibliography{Tex/ref}





\end{document}
